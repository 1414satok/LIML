\documentclass[a4paper, 12pt]{ltjsarticle}
\usepackage{graphicx} % Required for inserting images
\usepackage{luatexja} % Required for Japanese text
\usepackage{mathrsfs}
\usepackage{hyperref}
\hypersetup{
    colorlinks=true,
    linkcolor=black,
    filecolor=black, 
    citecolor=black,
    urlcolor=black
    }
\usepackage{booktabs}
\usepackage{amsmath}
\usepackage{rotating}
\usepackage{mathrsfs}
\usepackage{arydshln}
\usepackage{wrapfig}
\usepackage{caption}
\usepackage[subrefformat=parens]{subcaption}
\captionsetup{compatibility=false}
\usepackage{multirow}
\usepackage{setspace}
\usepackage{hyperref}
\usepackage[
  style=authoryear,
  sorting=nyt,
  isbn=false,
  url=false,
  doi=false,
  eprint=false,
  maxcitenames=2,
  maxbibnames=99,
  backend=biber
  ]{biblatex}
\renewcommand{\figurename}{図}
\renewcommand{\tablename}{表}
%addbibresource{input/.bib}
\renewbibmacro{in:}{}
\DeclareFieldFormat{pages}{#1}

\title{LIML推定量の導入}
\author{}
\date{}

\begin{document}
\maketitle
以下の同時方程式を考える。
\begin{align}
  \begin{cases}
    y_1 = \alpha_1 + \gamma_1 y_2 + \delta_1 x_1 + \epsilon_1 \\
    y_2 = \alpha_2 + \gamma_2 y_1 + \delta_2 x_2 + \epsilon_2
  \end{cases}
\end{align}

この式を$y_1, y_2$について解くと、
\begin{align}
  \begin{cases}
    y_1 = \frac{1}{1 - \gamma_1 \gamma_2} (\alpha_1 + \gamma_1 \alpha_2 + \delta_1 x_1 + \gamma_1 \delta_2 x_2 + \epsilon_1 + \gamma_1 \epsilon_2) \\
    y_2 = \frac{1}{1 - \gamma_1 \gamma_2} (\alpha_2 + \gamma_2 \alpha_1 + \gamma_2 \delta_1 x_1 + \delta_2 x_2 + \epsilon_2 + \gamma_2 \epsilon_1)
  \end{cases}
\end{align}
となる。さらに、行列を用いて表すと、
\begin{align}
  \vec{y}=\begin{pmatrix}
    \lambda\\
    \Gamma
  \end{pmatrix}X + e
\end{align}
where
\begin{eqnarray*}
  \vec{y}&=&\begin{pmatrix}
    y_1\\
    y_2
  \end{pmatrix}\\
  \lambda&=&\begin{pmatrix}
    \frac{\alpha_1+\gamma_1 \alpha_2}{1-\gamma_1 \gamma_2}&
    \frac{\delta_1}{1-\gamma_1 \gamma_2} &
    \frac{\gamma_1 \delta_2}{1-\gamma_1 \gamma_2}
  \end{pmatrix}\\
  \Gamma&=&\begin{pmatrix}
    \frac{\alpha_2+\gamma_2 \alpha_1}{1-\gamma_1 \gamma_2}&
    \frac{\gamma_2 \delta_1}{1-\gamma_1 \gamma_2} &
    \frac{\delta_2}{1-\gamma_1 \gamma_2}
  \end{pmatrix}\\
  X&=&\begin{pmatrix}
    1&x_1&x_2
  \end{pmatrix}^{\prime}\\
  e&=&\begin{pmatrix}
    e_1\\
    e_2
  \end{pmatrix}= \begin{pmatrix}
    \epsilon_1 + \gamma_1 \epsilon_2\\
    \epsilon_2 + \gamma_2 \epsilon_1
  \end{pmatrix}
\end{eqnarray*}
これをさらに
\begin{align}
  \vec{y}=\Pi_0^{\prime}+\Pi_1^{\prime}x_1+\Pi_2^{\prime}x_2+e
\end{align}
where
\begin{eqnarray*}
  \Pi_0&=&\begin{pmatrix}
    \frac{\alpha_1+\gamma_1 \alpha_2}{1-\gamma_1 \gamma_2}&
    \frac{\alpha_2+\gamma_2 \alpha_1}{1-\gamma_1 \gamma_2}
  \end{pmatrix}\\
  \Pi_1&=&\begin{pmatrix}
    \frac{\delta_1}{1-\gamma_1 \gamma_2}&
    \frac{\gamma_2 \delta_1}{1-\gamma_1 \gamma_2}
  \end{pmatrix}\\
  \Pi_2&=&\begin{pmatrix}
    \frac{\gamma_1 \delta_2}{1-\gamma_1 \gamma_2}&
    \frac{\delta_2}{1-\gamma_1 \gamma_2}
  \end{pmatrix}
\end{eqnarray*}
と書き直す。このとき、$\Pi_2$に対して
\begin{align}
  \Pi_2\gamma=0\qquad\text{where}\quad\gamma=\begin{pmatrix}
    1\\
    -\gamma_1
  \end{pmatrix}
\end{align}
が成り立つ。

\end{document}